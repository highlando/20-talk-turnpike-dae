\documentclass[xcolor=dvipsnames,10pt,hyperref={breaklinks=true}]{beamer}
\useoutertheme{miniframes}

\input{def}

\usetheme{sthlm}
\usepackage{times,bm,amsfonts,amsmath,amssymb,amsthm,tikz,fancybox} 
\usepackage{dsfont}
\usepackage{xspace}
\usepackage[utf8]{inputenc}
\usepackage[english]{babel}
\usepackage{hyperref}
\usepackage{animate}
\usepackage{calc}
\usepackage{multicol}
\usepackage{dcolumn}
\usepackage[babel=true,kerning=true]{microtype}
\usepackage[T1]{fontenc} 
\usepackage{color, colortbl}
\usepackage{beamerthemesplit}
\usepackage{mathrsfs} 
\usepackage{xcolor}
\usepackage[english]{varioref}
\usepackage{subfig}
\usepackage{pgfplots}
\usepackage{pgfplotstable}
\usepackage{tcolorbox}
\usetikzlibrary{matrix,external,fit}

\setbeamercovered{dynamic}
\setbeamertemplate{blocks}[shadow=true]
\setbeamertemplate{footline}[frame number]
\setbeamertemplate{section in toc}[mysections numbered]
\addtobeamertemplate{block begin}{\setlength\abovedisplayskip{0pt}\setlength\belowdisplayskip{0pt}}

\DisableLigatures{encoding = *, family = * }

\title{Turnpike in \\ Linear Systems Theory}
\subtitle{Old Results and New Formulas}
\date{\small{\jobname}}
\author{Jan Heiland \\ \scriptsize DeustoTech, Universidad de Deusto, Bilbao, Spain} 
\date{February 25, 2020}

\begin{document}
	
%%%%%%%%%%%%%%%%%%%%%%%%%%%%%%%%%%%%%%%%%%%%%%%%%%%%%%%%%%%%%%%%%%%%%%%%%%%%%%%%%%%%%%%%%
%%% FRAME	
%%%%%%%%%%%%%%%%%%%%%%%%%%%%%%%%%%%%%%%%%%%%%%%%%%%%%%%%%%%%%%%%%%%%%%%%%%%%%%%%%%%%%%%%%
\begin{frame}[plain]
\maketitle
\end{frame}

%%%%%%%%%%%%%%%%%%%%%%%%%%%%%%%%%%%%%%%%%%%%%%%%%%%%%%%%%%%%%%%%%%%%%%%%%%%%%%%%%%%%%%%%%
%%% FRAME	
%%%%%%%%%%%%%%%%%%%%%%%%%%%%%%%%%%%%%%%%%%%%%%%%%%%%%%%%%%%%%%%%%%%%%%%%%%%%%%%%%%%%%%%%%

\begin{frame}[b]
\frametitle{Old Results}
Consider, for $t_1>0$,
\begin{equation*}
  \int_{0}^{t_1} \|Cx(s)-y_c\|^2+ \|u(s)\|^2\inva s + \|F x(t_1)-y_e\|^2
  \to \min
\end{equation*}
subject to
\begin{equation*}
  \dot x(t) = Ax(t) + Bu(t), \quad x(0)=x_0.
\end{equation*}

\begin{theorem}[Callier/Winkin/Willems'94 with $y_c=0$ and $y_e=0$]
  Let $(A,B)$ be stabilizable and let
  \begin{equation}
    A^*X+XA-XBB^*X+C^*C=0 \tag{ARE}
  \end{equation}
  have a unique stabilizing solution $\PP$. Then:
  \begin{itemize}
    \item If $P(t) \to \PP$ as $t_1\to \infty$, then, for some $\lambda>0$,
      \begin{equation*}
        \|x(t) - \mxp{t(A-BB^*\PP)}x_0\| \leq C\mxp{-(t_1-t)\lambda}.
      \end{equation*}
    \item The solution to the differential Riccati equation $P$ converges to
      $\PP$ as $t_1\to \infty$ \textbf{if, and only if}, the nullspace of $F^*F$ and the
      undetectable subspace of $(C,A)$ have no intersection.
  \end{itemize}
\end{theorem}

\end{frame}


\begin{frame}
  \frametitle{Numerical Example}
\uncover<2>{
  \begin{minipage}[t]{0.28\linewidth}
    \begin{center}
    $F\perp C$:\\[.05in]
    \hline
    \input{pics/riccatientries.tex}
    % \hlinpic
    \input{pics/optistate.tex}
    % \hlinpic
    \input{pics/optiout.tex}
  \end{center}
  \vfill
  \end{minipage}
  \begin{minipage}[t]{0.28\linewidth}
    \begin{center}
      $F=C$:\\[.05in]
    \hline
    \input{pics/fiscriccatientries.tex}
    % \hlinpic
    \input{pics/fiscoptistate.tex}
    % \hlinpic
    \input{pics/fiscoptiout.tex}
  \end{center}
  \vfill
  \end{minipage}
}
  \begin{minipage}[t]{0.4\linewidth}
  \begin{itemize}
    \item System
  \begin{equation*}
    \begin{split}
    A &= 
    \begin{bmatrix}
      2 & 0 \\ 0 & -1 
    \end{bmatrix}\\
    B &= 
    \begin{bmatrix}
      1 \\ 1 
    \end{bmatrix}\\
    C &= 
    \begin{bmatrix}
      0 & 2 
    \end{bmatrix}
  \end{split}
  \end{equation*}
    \item Not detectable but
    \item (ARE) has a stabilizing solution.
    \item 
  Simulations with
  \begin{equation*}
    \begin{split}
    x_0=
    \begin{bmatrix}
      1 \\ 1 
    \end{bmatrix}, \\y_c=0, \quad y_e=1.
  \end{split}
  \end{equation*}
  \end{itemize}
  \vfill
  \end{minipage}%
\end{frame}

%%%%%%%%%%%%%%%%%%%%%%%%%%%%%%%%%%%%%%%%%%%%%%%%%%%%%%%%%%%%%%%%%%%%%%%%%%%%%%%%%%%%%%%%%
%%% FRAME	
%%%%%%%%%%%%%%%%%%%%%%%%%%%%%%%%%%%%%%%%%%%%%%%%%%%%%%%%%%%%%%%%%%%%%%%%%%%%%%%%%%%%%%%%%

\begin{frame}
  \frametitle{New Formulas for $y_e$ and $y_c$ not zero}
Main tool: an explicit formula\footnote{Not unknown before, but explicitly
derived in Behr/Benner/Heiland'19} for $U$ that solves
\begin{equation*}
  \dot U(t) = (A-BB^*P(t))U(t), \quad U(t_1)=I,
\end{equation*}
and that defines the state transition like
\begin{align*}
  x(t) &= U(t)U(t_0)^{-1}x_0 + \int_0^t U(t)U(s)^{-1}w(s)\inva s,
\end{align*}
where $w$ is the feedforward (the affine) part that satisfies
\begin{equation*}
  -\dot w(t) = (A^*-P(t)BB^*)w(t), \quad w(t_1)=-F^*y_e,
\end{equation*}
or
\begin{equation*}
  w(t):=-U(t)^{-*}U(t_1)^*F^*y_e + \int_{t_1}^tU(t)^{-*}U(s)^*C^*y_c.
\end{equation*}
\end{frame}


\begin{frame}
  \frametitle{New Formulas for $y_e$ and $y_c$ not zero}

  With this $U$, we find that
\begin{align*}
  x(t) &= U(t)U(t_0)^{-1}x_0 + \int_0^T U(t)U(s)^{-1}w(s)\inva s \\
       &= \quad \quad ...\\
       &= U(t)U(t_0)^{-1}x_0 + \AP^{-1}BB^*\AP^{-*}C^*y_c + G(t,t_1),
\end{align*}

\begin{itemize}
  \item where $U(t)U(t_0)^{-1}x_0$ is the solution to the problem with $y_c$,
    $y_e=0$,
  \item where
    \begin{equation*}
    \AP^{-1}BB^*\AP^{-*}C^*y_c = (A-BB^*\PP)^{-1}BB^*(A-BB^*\PP)^{-*}C^*y_c
  \end{equation*}
    is the steady state optimal solution,
  \item and where, for some constant $M$ independent of $t_1$, 
    \begin{equation*}
    \|G(t,t_1)\| \leq
    M\mxp{-(t_1-t)\lambda}+M\mxp{-(t_1-t)2\lambda}+M\mxp{-(t_1-t)3\lambda} .
  \end{equation*}
\end{itemize}

\end{frame}

\begin{frame}
\frametitle{Next steps and discussion}

\begin{itemize}
  \item The transition maps also cover the time-dependent case...
  \item Transition maps (and turnpike) for LQR Problems with DAE constraints
  \begin{equation*}
    E \dot x = Ax + Bu, \quad Ex(0)=Ex_0.
  \end{equation*}
  \begin{itemize}
    \item $E$ is nonsingular $\leftarrow$ singular control.
    \item Nonsymmetric Riccati equations.
  \end{itemize}
  \item Transition maps (and turnpike) for $\infty$-dimensional systems.
\end{itemize}

\end{frame}

\begin{frame}
  \frametitle{References}

 \vfill

 Behr, M.; Benner, P. \& Heiland, J. \emph{Invariant Galerkin Ansatz Spaces and
   Davison-Maki Methods for the Numerical Solution of differential Riccati
 Equations} arXiv e-prints, 2019, arXiv:1910.13362 

 \vfill

 \hline

 \vfill

 Callier, F. M.; Winkin, J. \& Willems, J. L.  \emph{Convergence of the
   time-invariant Riccati differential equation and LQ-problem: mechanisms of
 attraction} International Journal of Control, 1994, 59, 983-1000 

 \vfill

\end{frame}

%%%%%%%%%%%%%%%%%%%%%%%%%%%%%%%%%%%%%%%%%%%%%%%%%%%%%%%%%%%%%%%%%%%%%%%%%%%%%%%%%%%%%%%%%
%%% FRAME	
%%%%%%%%%%%%%%%%%%%%%%%%%%%%%%%%%%%%%%%%%%%%%%%%%%%%%%%%%%%%%%%%%%%%%%%%%%%%%%%%%%%%%%%%%

\begin{frame}
\begin{large}
	\begin{center}
		\color{lapislazuli}\fontfamily{ugq}\selectfont THANK YOU FOR YOUR ATTENTION!\color{black}
	\end{center}
\end{large}

\begin{tcolorbox}[width=\textwidth,colframe=darklavender,colback=white]    
	\small{This project has received funding from the European Research Council (ERC) under the European Union’s Horizon 2020 research and innovation program (grant agreement No 694126-DYCON).}
\end{tcolorbox} 
% \vspace{1cm}   %\vspace{0.5cm}
% \begin{center} 
% 	\begin{figure}
% 		\centering
% 		\begin{minipage}{0.25\linewidth}
% 			\includegraphics[scale=0.15]{./logos/logo-ce}
% 		\end{minipage}
% 	    \hspace{1cm}
% 		\begin{minipage}{0.25\linewidth}
% 			\includegraphics[scale=0.07]{./logos/logo-ERC}
% 		\end{minipage}
% 	\end{figure} 
% \end{center}	
% 
% \begin{center}
% 	\begin{figure}
% 		\centering
% 	    \begin{minipage}{0.25\linewidth}
% 			\includegraphics[scale=0.05]{./logos/logo-DyconERC}
% 		\end{minipage}
% 		\hspace{0.5cm}
% 		\begin{minipage}{0.25\linewidth}
% 			\includegraphics[scale=0.1]{./logos/logo-deustotech}
% 		\end{minipage}
% 		\hspace{0.45cm}
% 		\begin{minipage}{0.25\linewidth}
% 			\includegraphics[scale=0.05]{./logos/logo-CCM}
% 		\end{minipage}
% 	\end{figure}
% \end{center}
\end{frame}
\end{document}
